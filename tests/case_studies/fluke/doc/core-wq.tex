\subsection{Wait Queue Model}

%\xxx{describe wait queues here - how they're implemented,
% what the essential abstraction is, i.e., do we need a guaranteed
% ordering etc., discuss multiprocessor, fairness, starvation}

A wait queue is a data structure used to represent a FIFO list of
threads waiting for a particular event. This is used to keep track 
of the list of threads waiting on a condition variable. 

% We have two implementations of wait queues XXX -- forget it for now --Pat

% -- old paragraph was
% Wait Queues are implemented as a doubly linked list.  The Wait Queue
% itself contains a ``pointer'' to the head and tail.  (The pointer is
% just the id of the thread.)  Additionally, each thread has storage for
% a ``next'' and ``prev'' pointer.  Since it can by definition only ever
% be on one wait queue at a time, this is sufficient.

% new paragraph is
Wait Queues are implemented as a doubly linked list of threads where the 
Thread id are used as pointers. Each thread object contains one ``next''
and one ``prev'' pointer -- this is sufficient since
a thread can only be on one wait queue at a time.
% and goes until here
The interface provided by our model is exactly the same as that
provided by the kernel's C implementation. In addition, the Promela
implementation allows to remove an arbitrary thread from the
wait queue.
 
