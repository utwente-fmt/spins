% %%%%%%%%%%%%%%%%%%%%%%%%%%%%%%%
%
% CS 611 Project Report
%
% %%%%%%%%%%%%%%%%%%%%%%%%%%%%%%%

% Document setup info
\documentstyle{article}
     
%%%%%%%
%%%%%%% Some standard \def's 

% We don't need no steenkin' equations - just gimme a working underscore!
\catcode`\_=\active 

\long\def\com#1{}
\long\def\xxx#1{{\em {\bf Fix: } #1}}

%%%%%%%
%%%%%%%

\topmargin 0pt
\advance \topmargin by -\headheight
\advance \topmargin by -\headsep
\textheight 8.9in
\oddsidemargin 0.3in  
\evensidemargin \oddsidemargin
\marginparwidth 0.5in
\textwidth 6in

%% Heh.. Not sure what this does... but it allows me to include postscript(?)
%\input{psfig}

\title{\bf \Large Fluke IPC Verification Project Report}

\author{Patrick Tullmann~~~~Godmar Back~~~~Ajay Chitturi~~~~\\
        John McCorquodale~~~~Jeff Turner\\[2ex]
        {\tt veripc@jensen.cs.utah.edu}
        }

\date{November 27, 1996}

%BEGIN DOCUMENT--------------------
\begin{document}

%HEADER
\maketitle

%%%%%%%
%
% abstract follows
%

\begin{abstract}
    We have formally described and modeled the reliable Interprocess 
    Communication (IPC) as implemented in the Fluke operating 
    system\cite{ford+:osdi96}. With this model we were able to verify
    certain properties, such as freedom of deadlock, under a number of
    test scenarios.  We have also provided a base of documentation and
    infrastructure for formal verification of other areas of Fluke.
    Our model tries to follow closely the actual C implementation
    of the IPC system, allowing us to make strong claims about 
    the actual system, as opposed to claims simply about the design.
\end{abstract}

\tableofcontents
\newpage

%%%%%%% 
% give an introduction
% intro.tex - What we did and did not do, and a hint of why, with forward refs
%------------------------------------------------------------------------------

\section{Introduction}
\label{intro}

%%% Godmar/Pat draft 1:
%%% McQ draft 1: 
%%% Godmar revisions
%%% McQ revisions
%%% Godmar revisions 2 (mostly threw stuff out)

Fluke is a microkernel implemented in an object-based spirit.
Fluke operations are made available as methods on a collection of
primitive {\it Fluke objects} such as threads, ports, memory regions,
mutexes, condition variables, and others.  Fluke's high-level organizing
pattern is that of {\em recursive virtual machines\/}
or {\em nested processes\/}\cite{ford+:osdi96}. Nested processes  
communicate via the low-level Fluke inter-thread communications facility, 
called {\it Fluke IPC}, to provide a virtual machine abstraction.

Fluke allows the kernel state of an object to be exported {\em at any time}
\cite{Tullmann+:iwooos96}. This even holds for long-running kernel operations
in which an object may be involved. Therefore, these operations must
be decomposable into smaller units that leave each object in a
consistent state which can be exported. A checkpointer may extract
an object's state, pickle it and use the exported state to reconstruct
and restart these objects at a later time.

For instance, the Fluke IPC mechanism is designed to allow suspension 
and resumption of long-lived IPC operations.  The operations are broken
into smaller pieces. Objects enter and leave these smaller primitives
in a consistent state. Since the Fluke thread objects of the participating
threads are used to keep track of the IPC operation's progress, 
these consistent states must be valid restart-points for these threads.

Our work thus far has focused on model-checking the Fluke IPC subsystem,
in particular the Fluke {\it reliable IPC} mechanism, which is the most
complex, subtle and important of the three flavors of Fluke IPC.

The achievements of this effort are threefold.  First, we have
reverse-engineered a descriptive model of the Fluke IPC subsystem from
the current C implementation.  Second, we used this descriptive 
model to develop a checkable model of the reliable IPC subsystem in
Promela \cite{Spin-book} which we present in sections \ref{ipc} and
\ref{core}.  Lastly, we have begun evaluating the performance of this
model under various realistic scenarios with encouraging results, as
presented in section \ref{results}.  This section we also
describe our experiences with SPIN and list some perceived shortcomings.
We conclude with an evaluation of the prospects of using
the SPIN model-checker as the basis for a larger-scale formal
verification effort of the Fluke kernel.



% intro to fluke IPC and desc of modelling it

\section{IPC and Threads}
\label{ipc}

\subsection{Overview of Fluke IPC}

This section gives the basics of the Fluke reliable IPC mechanism.
Two other forms of IPC in Fluke, One-way and Idempotent, are
described in \cite{Flukedocs:96}.

{\em Reliable IPC} is the mechanism Fluke provides
for general, high-performance, reliable, protected communication.
Reliable IPC in Fluke operates in a stream-oriented model,
in which two threads connect to each other directly
and can send data to each other as if through a pipe.
These connections are symmetrical but half-duplex,
meaning that data can only flow in one direction at a time.
However, the direction of data flow can be reversed
without destroying the connection.

Any Fluke thread can be involved in two reliable IPC connections at once:
any thread may have both a connection to a {\em client}
thread--a thread that ``called'' it,
as well as a connection to a {\em server} 
thread--a thread that it has ``called''.

Either side of a reliable IPC connection
may be held up arbitrarily long while waiting for the other side.
%e.g. because of page faults in the other task that need to be handled.
Thus, there are no guarantees regarding the ``promptness''
with which a server thread can perform its work and disconnect.
A server thread is bound to a particular client thread
for the duration of the IPC operation.

Communication within the kernel is based on unilateral
synchronization--interacting threads are required to cooperate with
each other.
Threads effectively ``ask'' other threads to complete an operation
on their behalf, blocking until the result is available.  For example,
during a standard reliable IPC operation, the server thread will
block waiting for a client to show up.  It blocks in 
a state labeled as ``waiting to receive''.  The client will 
grab the server and set up the connection.  Once the connection is
established, it changes the {\em server's} state to reflect an
active connection (just in case they get disconnected.) and then
starts transferring data to the server.  When the transfer is 
completed, the client puts the server back on the ready queue so
that it may run.  When the server wakes up it will restart as
if it had been doing a receive operation. The client will go
to sleep in a state indicating that it has finished a send.  The
server wakes up and notices the client is in the right state,
and then restarts the client, and returns to the user itself.

It is important to note that while the 
thread is blocked in a specific ``wait state'' it will not
be restarted until its counterpart in the reliable IPC operation
restarts it.

The IPC implementation uses {\it wait state values} to keep track of
the state in which a particular thread is waiting. For example, if a
thread is currently a server (in an IPC connection) and is waiting to
send data, its wait state value will be {\tt WAIT_IPC_SRV_SEND}.
The set of wait states forms a tree whose
leaves are single states. The leaves are represented in a binary
encoding such that a single bit test suffices to determine whether
a state belongs to a certain subtree, i.e. is a member of a subset of 
states.   See Figure~\ref{waitStates.fig} for the tree of valid
wait states.

\begin{figure}[th]
{\small
\begin{verbatim}
WAIT_NONE
WAIT_ANYTHING ->
     WAIT_IPC_PICKLE
     WAIT_STOPPED
     WAIT_CANCELABLE ->
             WAIT_ON_COND
             WAIT_IPC_SRV_DSCN
             WAIT_IPC_SRV_RECEIVER ->
                     WAIT_IPC_SRV_RECV
                     WAIT_IPC_SRV_ASEND
             WAIT_IPC_SRV_SENDER ->
                     WAIT_IPC_SRV_ORECV
                     WAIT_IPC_SRV_SEND
             WAIT_IPC_CLI_DSCN
             WAIT_IPC_CLI_RECEIVER ->
                     WAIT_IPC_CLI_RECV
                     WAIT_IPC_CLI_ASEND
             WAIT_IPC_CLI_SENDER ->
                     WAIT_IPC_CLI_ORECV
                     WAIT_IPC_CLI_SEND
             WAIT_IPC_UNPICKLE
             WAIT_IPC_IDEMPOTENT ->
                     WAIT_IPC_EXCEPTION
                     WAIT_IPC_CALL
                     WAIT_IPC_SECURITY
\end{verbatim}
} %end \small
\caption{Valid wait states for blocked threads}
\label{waitStates.fig}
\end{figure}


% XXX keep?
\subsection{Promela Implementation of Fluke IPC}

The Promela model of the IPC code resembles the implementation very closely. 
% Pat, I don't like the phrase "have been abstracted" or even 
% "have been abstracted away"
Very few details have been discarded or replaced with implementations
of equivalent abstract semantics. 
The majority of functions which model the reliable IPC path are 
implemented by directly translating the corresponding C functions to 
Promela. (The C functions are described in some detail in the Appendix.)
For example, the following is a snippet of C that implements the
{\tt sys_ipc_client_connect_send_over_receive()} function--the
kernel interface for that function.

\begin{verbatim}
        struct s_thread * const client = current_thread();
        int rc;

        rc = ipc_client_1_connect_send(client,
                                       fluke_ipc_client_send_over_receive__entry);
        if (rc) return rc;

        return ipc_client_2_over_receive(client, WAIT_IPC_SRV_RECV);
\end{verbatim}

Differing naming conventions aside, it can be seen that the 
``equivalent'' Promela looks almost identical:

\begin{verbatim}
        ipcClient1ConnectSend(currentThread,
                              threadExcConnectArg(currentThread),
                              ENTRYPOINT_CLIENT_SEND_OVER_RECEIVE);
        if 
        :: rc==0 ->
             ipcClient2OverReceive(currentThread, WAIT_IPC_SRV_RECV);
             /* sets rc, RETURN rc */
        :: else ->
             /* RETURN non-zero rc */ skip;
        fi;
\end{verbatim}

Since we don't model ports and port sets, the id of the
server thread is passed as a parameter (this is accomplished by
{\tt threadExcConnectArg(currentThread)}).

% --this seems obsolete
% There was one area of the code that was easy and worthwhile to
% simplify.  The data transfer.

\subsubsection{Transfer of data}
In the C implementation, the function {\tt ipc_reliable_transfer()}
copies the data and references from the sender space to the receiver space.  
This functions implements scatter/gather buffers, the bookkeeping
needed for reference transfer, and other things -- all of which are
outside the scope of our verification. 

In our Promela implementation, we simplify the data transfer to its
essence. Only a single byte of payload (which is part of the thread's 
exception state) is transferred from the sender to the receiver. 
%This is made use of in assertions
%which check if the data received is the same as the data that has been
%sent.

The possibility of either the sender or the receiver page faulting
during the transfer is implemented using the non-determinism that
Promela provides.  Note that the payload is copied only in the case in
which there is no error.  Additionally, because transfer is isolated to 
a single function we are able to make the inclusion of page fault
a configurable option.


%\input{model-ipc.tex}

%
%
%

\subsection{Threads in Fluke}

Threads are a conventional execution context in Fluke.
They run in an address space and access system services through
a defined system call interface.  The interface is object-based.
Threads not only manipulate mutexes or other kernel objects
through that interface, but also each other. For example, a thread
may stop another thread simply by setting its state to {\tt STOPPED}.

% XXX threads have two halves, kernel and user mode

Threads in kernel space can directly access the state of other threads, 
provided they obey the locking protocol.  
Thus a client thread in an IPC operation
can directly manipulate the state of the server thread when sending
data to it.  Another example is when a thread needs to cancel another
thread -- stopping its current execution and making it roll back to a
clean entry point.  First it locks the thread object, then it sets the
target thread's ``cancel pending'' bit, and blocks.  When the target
thread undoes its current operation and enters the code to handle the
cancel, it will notice the blocked thread and wake it up.

\subsubsection{Thread State}
There are four relevant sections of the Fluke thread state to our
model, each of which is protected by an associated mutex.  First 
is the generic state that all objects in Fluke have, this identifies
the type of the object, whether it is active, etc.  Second is
the state associated with any outstanding IPC connections.  Next
is the state describing a blocked thread (the wait state).  Last
is the mis-named exception state which contains the state of the
thread's user-kernel boundary. 

% XXX say something about the exc.ip !

\subsubsection{Scheduling}
Threads may be blocked, ready or running. At any time, at most
$n$ threads may be running, where $n$ is the number of processors
in the system. Unblocking a thread is accomplished by putting it
on the ready queue. A dispatcher is then invoked which will
eventually schedule that thread.  The kernel is designed to
run on SMP machines and have multiple threads executing
concurrently.

\subsection{Promela Implementation of Threads}

Threads are modeled using Promela's {\it proctype} construct.
Each thread is represented by a proctype which calls a
particular function (implemented using a macro) 
corresponding to the kernel entrypoint. 
All thread state is stored in a global array of
structures, one element per thread.
The specific state modeled is described below.

Initially we experimented with using two separate 
proctypes to encapsulate the state and control portions of the
thread. We expected this to hide a lot of information in local
variables and thereby enable SPIN to optimize better. However, this
approach involved a lot of communication between the state and control
proctypes and so disabled extensive use of SPIN's {\it d_step}
construct which allows {\em us} to optimize the control states.
Since specifying liveness conditions using SPIN's never
automaton requires access to global variables, we decided to take the
approach of using one proctype per thread and made all thread
state global.

\subsubsection{Thread State}
In our paradigm of translation of C into Promela, we kept much of the
structure of the thread state in our models.  Each thread is comprised
of some 4 parts.  There are is the wait queue information (next, and
prev pointers); the IPC state information; information about the
thread's current wait state; and the exception information.  

The IPC state contains both explicit ``pointers'' to the active client
and server threads, and models of pickled ``links''.  It includes some
flags of the current state of the IPC.  The wait state includes a
value indicating the current state the thread is in, if it is blocked,
the condition variable it is blocked on (if it is blocked on one.)  The
cancel pending flag and the resume return code to use after
unblocking.  The somewhat mis-named exception state includes all of
the data used for communicating with user mode--additionally it stores
the ``instruction pointer'' reference used in rollback--See the
Kernel Entry Layer section below.

The only state which does not have a direct correspondence from the 
C code to the Promela code is the rendezvous point and
the data payload, both in the IPC section of the thread state.

% -- this has already been said
% The rendezvous is used in our simplification of establishing a
% connection (see the section on Ports and Port Sets below.)  The
% data payload is used in the simulated data transfer--we restrict
% the data transferred to a single byte.

\subsubsection{Scheduling}
We make use of SPIN's interleaving of the proctypes to model the
thread scheduling. So by default any of the threads is runnable. If a
thread has to {\it block} (for example while waiting on a condition 
variable), it
sets a particular bit in its global state and waits for some other
thread to reset this bit. The thread which resets the bit
{\it unblocks} this thread. 

Since SPIN tries all possible interleavings of the code during
verification, we model an SMP scenario in which each of the runnable
threads gets a processor to run. This also means that the results
obtained by this approach do not depend on any scheduling algorithm. 
Modeling the single-processor case would actually be more difficult(!)\@
since we would have to come up with a mechanism where by only one thread
could run until it explicitly gave up control of the processor.


%\input{model-threads.tex}


% desc of low-level objects and implementation
\section{Fluke Core Models}
\label{core}

While evolving our current Promela model of Fluke
IPC we have developed a support infrastructure which includes Promela
implementations of many low-level Fluke objects used during Fluke IPC 
sessions. We call this support infrastructure the
{\it Core Models}.  These models model such Fluke entities as mutexes,
condition variables, wait queues, links, ports and port sets, and the
Fluke kernel entry layer (which embodies behavior intimately related to
the cancellability of IPC operations).

Implementation details and the design rationale of these models are
presented here.  Where appropriate, the implementation of these models is
quite faithful to the C-language source from which they are derived.  When
particular divergence from the C-code has occurred, we note the underlying
motivation and discuss the semantic similarities between the two
implementations.

Since our model is heavily based on the actual implementation, we have
a high measure of confidence that the results obtained from model
checking are in fact applicable to the real system.

\subsection{Translation Artifacts and the Sliding {\tt d_step} Technique}

Since Promela does not support language-level procedures or named blocks,
we were forced to rely on C preprocessor macros to give us the ability to
name code segments and provide a function-like decomposition of source
code.  Since these macros are not lexical scopes, and many instances of
the same macro can be instantiated in a single lexical scope (proctype),
these macros cannot contain variable declarations, labels or other symbol
declarations or duplicate symbol declaration will occur.  Thus, most macros
assume the presence of a local variable called {\tt rc} which
is used to hold the ``return value'' from the ``function'' that the macro
embodies.  This variable can be interrogated by the user of the macro.

The Core Models are designed to allow the use of a
technique we call the {\it sliding {\tt d_step}}.  When working with
a model as large as the Fluke IPC model, the problem of state-space
explosion is an omnipresent threat.  Since the Core Models are largely
self-contained and contain no external references, individual Core
Models (or groups of them) can be verified together, and then 
their ``public methods'' can be encased with the SPIN/Promela 
{\tt d_step} statement, which will avoid an overall increase in 
state-space. 
%% "can be kept at the edge of manageability" is too wordy... 
%% but I'll leave it in since I can't immediately come up with something better
The positioning of these {\tt d_steps} is quite flexible due to the
hierarchical and rigidly encapsulated structure of our models.  Thus,
problems can be kept at the edge of manageability, to verify the Fluke
implementation as rigorously as possible using the hardware available.

\subsection{Mutex Model}
\label{mutex-model}

{\it Mutexes} are a mechanism for
achieving mutual exclusion among a set of concurrently executing threads.  
A mutex is an object that can be in one of two states, {\it locked} and 
{\it unlocked}.  Mutex objects must support {\it lock} and {\it unlock},
which move the mutex from the unlocked state to the locked state, and
vice versa. Mutual exclusion is guaranteed since the lock and unlock
operations are atomic and only one contender will successfully acquire
the lock. Attempts to lock a mutex that is already locked by some 
other thread will cause the calling thread to wait until the mutex is 
unlocked.  

\paragraph{Fluke {\tt mutex} Objects}

The Fluke implementation of mutexes uses Fluke wait queues to keep track
of threads waiting to acquire a particular mutex.

\paragraph{The Promela {\tt Mutex} Type}

Since Promela has builtin atomicity support via the {\tt atomic}
and {\tt d_step} constructs, it was not necessary for us to build our
model Mutexes on top of our model WaitQueues. 
Rather, one bit is sufficient to represent the state of a mutex.
We call Mutexes with one state bit {\it simple} mutexes. They are
implemented in {\tt src/fluke/Mutex-simple.pr}.

However, there are many useful properties that one may wish to verify
about mutexes and how they are used by other code.  For example, it may
be desirable to assert that the same thread that locked a mutex must unlock
it. Also, we might wish to
ensure that the {\it same} thread does not try to lock a mutex it already
holds since this action would result in deadlock.  
We have a second, more heavyweight,
implementation of Promela Mutexes, called {\it safe} mutexes that address
these concerns and aggressively make sanity assertions.  This implementation
can be found in {\tt src/fluke/Mutex-safe.pr}.

Both implementations of Promela Mutexes support the same set of functions.
The intention is that {\tt Mutex-safe.pr} will be used to verify the
appropriate use of Mutexes in a system, and then {\tt Mutex-simple.pr}
will be substituted when the whole system is tested, to minimize state
vector length.


\subsection{Wait Queue Model}

%\xxx{describe wait queues here - how they're implemented,
% what the essential abstraction is, i.e., do we need a guaranteed
% ordering etc., discuss multiprocessor, fairness, starvation}

A wait queue is a data structure used to represent a FIFO list of
threads waiting for a particular event. This is used to keep track 
of the list of threads waiting on a condition variable. 

% We have two implementations of wait queues XXX -- forget it for now --Pat

% -- old paragraph was
% Wait Queues are implemented as a doubly linked list.  The Wait Queue
% itself contains a ``pointer'' to the head and tail.  (The pointer is
% just the id of the thread.)  Additionally, each thread has storage for
% a ``next'' and ``prev'' pointer.  Since it can by definition only ever
% be on one wait queue at a time, this is sufficient.

% new paragraph is
Wait Queues are implemented as a doubly linked list of threads where the 
Thread id are used as pointers. Each thread object contains one ``next''
and one ``prev'' pointer -- this is sufficient since
a thread can only be on one wait queue at a time.
% and goes until here
The interface provided by our model is exactly the same as that
provided by the kernel's C implementation. In addition, the Promela
implementation allows to remove an arbitrary thread from the
wait queue.
 


\subsection{Condition Variable Model}

The C implementation of condition variables
makes use of POSIX-like condition
variables. In addition the interface provides a call which allows a
thread to place another (captured) thread asleep on a condition
variable. Unlike POSIX conditions variables, the associated mutex
is not re-acquired before returning from a wait operation.

A condition variable is modeled using a {\bf wait queue} of the
threads waiting on the condition variable. There is no extra state
required to be maintained as the wait queues implement all the
required features.

\com {
% Not any more, Pat ``fixed'' this.
One change had to be made in the implementation of the condWait()
function because of the way our wait queues were implemented. The
Promela implementation of waitQueueAdd() blocks where as the C
implementation does not. So we had to place some of the statements that
should actually have been executed after the waitQueueAdd(), before it
and place the whole sequence in an atomic block. This leads to a
slight difference in semantics whereby some of the interesting
interleavings are ignored.
}


\subsection{Link Model}

Existing, usable IPC connections can be in two states: active, and
pickled.  Active connections are internally represented as a direct
pointer from on thread to another. (ie, {\tt client\verb\->\ipc_state.server}
is the address of the server's s_thread_t object).  Pickled connections
are represented as {\em links}.  One or the other is accurate, never
both (the non-active link is 0.)  A direct pointer can be
thought of as a 'cached' value of the link.

The kernel provides {\bf links} which represent 
inter-object references.  They are used for
encapsulating pointers and for reference counting.
Initially we will model
these as simply as possible; we may add
verification of the reference counting later.



\subsection{Port and Port Set Models}

Ports are Fluke objects representing targets for IPC requests.
Ports might represent CORBA or MOM objects, for example,
or in Fluke implementations implementing full protection,
they may be used as secure capabilities.
A single server may maintain a number of ports,
each distinguishing a particular object, client,
or object/client pair.

Port sets are objects that act as a rendezvous point
between client threads attempting to perform IPC to a port
and server threads waiting for incoming IPC requests on one or more ports.
Multiple ports can be attached to a single port set,
and multiple server threads can wait for incoming requests
on a single port set at once.
This allows a single ``pool'' of service threads
to service incoming IPC requests from many ports
(i.e. requests associated with many ``objects'' the server exports).

Our model assumes that threads to connect
to for IPC have been ``found'' (the purpose of ports, port sets, and
references). Thus, models of ports or port sets were not
implemented in detail.  The only relevant interfaces used are 
{\tt s_pset_wait_receive()} by servers and {\tt s_port_capture()} by
clients.  The semantics of these operations have been distilled from
the C code.  

% This stuff below has to be made more clear
\com{
Specifically, a thread will block if its ``other half''
hasn't arrived yet.  It will check its cancel pending bit before
blocking.  It will block on a condition variable, in the WAIT_ON_COND
wait state.  When it is awoken it will, by default, return a
KR_RESTART code to its caller.  Additionally, the server thread {\em
always} blocks.  If it arrives and a client is waiting for it, then it
will awaken the client and then block.  In this way it is implied that
the client has the server ``captured'' when it returns from 
{\tt s_port_capture().}}

In our Promela implementation, we model these two functions to implement
the interface and make use of as little state as possible. The
rendezvous point is actually implemented as a bit in the server's
state. If the rendezvous bit is set, it means that either the client
or the server is waiting for a connection.
So when a client wants to connect to a server, it checks the
rendezvous bit to see if the server is waiting for a connection. If
the server is waiting, the server is captured and the connection
is made. If the server is not waiting for a connection, the client
sets the rendezvous 
bit and waits for the server to notice it. When a server is waiting
for an IPC  connection, it checks the rendezvous bit and if a client
is waiting, it makes the connection with the client. Otherwise the
server sets the rendezvous bit and waits for the client. In either
case the server blocks and is captured by the client which eventually
establishes a connection with this server.

% Do we need to say something about the non-cancelability of the wait 
% out here ?

It is interesting to note that the majority of the bugs in our model
came through this interface.  Since we strayed from the actual
implementation of the ports we missed subtle nuances like the server
always being blocked, etc. 



\subsection{Kernel Entry Layer Model}

A thread performing kernel operations is ``controlled'' by the kernel
entry point and the return code. 
All kernel entry points are ``clean.''  A thread that gets canceled
will never reveal itself to be in the middle of a kernel operation;
it will always appear to the canceling thread that it is about to
begin a kernel operation, or that it is in user mode.  To prevent
long or complicated kernel functions from becoming a bottleneck, these
operations are broken into sub-sequences.  

For example the long 
{\tt fluke_ipc_client_connect_send_over_receive()} operation that
connects to a server, sends a request, waits for a reply and receives
it into a buffer, can take quite a while.  As portions of the
operation are completed the kernel entrypoint is advanced.  So, in 
this example, after the ``connect'' phase is completed, the kernel
sets the thread's entrypoint to be {\tt
fluke_ipc_send_over_receive().}

\begin{figure}
{\small
\begin{enumerate}
\item {\tt KR_USER_EXCEPTION} A processor exception occurred which
should be blamed on the user (e.g. because the exception was generated
while accessing user space). The current thread's exception_state
contains the details.

\item {\tt KR_PAGE_FAULT} Page fault occurred.
This gets turned into a real KR_USER_EXCEPTION by the kentry layer
if the page fault cannot be resolved in the kernel
and no appropriate region keeper can be found to handle the fault

\item {\tt KR_CANCEL} Another thread is trying to manipulate us and
has asynchronously canceled us,  e.g. due to thread_interrupt(),
thread_get_state(), thread_set_state().

\item {\tt KR_NO_MEMORY} Ran out of kernel memory.

\item {\tt KR_RESTART} This return code indicates that we have context
switched due to a wait, and we need to restart execution in user mode
before doing anything else in case dependent things have changed.

\end{enumerate}
}
\caption{The set of return codes used within the kernel}
\label{ReturnCodes-fig}
\end{figure}

All of the major functions within the kernel return one of a small,
well defined set of error codes.  See Figure~\ref{ReturnCodes-fig}
for a complete list.
The return code is used to signal special conditions. By convention,
any function returning a non-zero return code signals that the
kernel operation in progress has been canceled or must be restarted.




% describe the one or two experiments that run
%
%
%

\section{Results}
\label{results}

In addition to the descriptive models, and a greater understanding of
the Fluke IPC code, this project generated quite a bit of
documentation of the kernel internals. We have static call traces of
the IPC system from the user-level entry point down into the core
kernel routines. We used fully cross-referenced html versions of the
kernel code~\cite{godmars-html}.

The most important results are those from the SPIN generated
verifications of various scenarios.  Because we modeled the kernel
entry layer accurately we are able to
generate simple test cases that are easy to code yet quite flexible.

A standard test case will enumerate the list of kernel entry-points it
plans to use. The preprocessor will ensure that only the need parts
of the Promela IPC implementation are included, dramatically reducing
compile time by omitting all unused code. One proctype is created for 
each user-level thread. These proctypes can then ``call'' user-level 
fluke functions by macro expansion. Consider the basic send-and-receive
test as an example: 

\begin{verbatim}
  do
  :: TRUE ->
       sendData = 42;
       flukeClientConnectSend(fluke, server, sendData);
       assert(rc == 0);

       flukeClientDisconnect(fluke);
       assert(rc == 0);
  od;
\end{verbatim}

The {\tt fluke} parameter is a handle to a proctype implementing the 
kernel.  All of our functions implicitly set a global variable 
{\tt rc}, Promela's {\tt assert()} construct is then used to verify
successful completion.

Certain simulation parameters are configurable through the use
of preprocessor constants before including the Promela files that
contain the definitions of the IPC model. At this point, there are three
parameters which can be set. First, one can select whether or not 
a simulated transfer can generate a page fault. Every transfer can 
cause a page fault in either the sender, or the receive, or it cannot
cause a page fault at all. This is controlled with 
{\tt INCLUDE_IPC_TRANSFER_FAULTS}.

Secondly, {\tt INCLUDE_IPC_PAYLOAD} determines whether IPC data payload
should be included in the simulation. With this option, a single byte
of data is transferred from the sender to the receiver. This lets us
assert that data hasn't been altered during the IPC transport.

Lastly, it is possible to choose between two distinct models of 
mutexes in Promela (see section~\ref{mutex-model}.) 
They differ in the number of checks and the amount of state 
associated with each.  The ``simple'' mutexes are a minimal 
implementation without any consistency checks. 
The ``safe'' implementation includes safety checks (via {\tt assert()})
and requires some extra state to be stored with each mutex. 
It is used if {\tt INCLUDE_MUTEX_SAFE} is defined.

%%
%% For each test:
%%   Basic desc.
%%   What it touches
%%   SPIN results
%%   Limitations.

%%%%%%%% Basic-send
\subsection{The Basic Send Test}

This test, {\tt src/test/Basic-send.pr}, is the first real test we
consider here.  It is is a very basic send-and-receive operation
involving two threads, a client and a server.  The client repeatedly
connects, sends a byte of data, and disconnects.  The server waits,
receives the byte of data, disconnects and then starts over.

%
% please allude earlier to the roll-back aspect
%
This tests the kernel entry layer, mutexes, the IPC connection
logic; it also tests a bit of the rollback, as threads effectively
rollback when moving from, for example, the ``wait'' to the
``receive'' in a ``wait_receive'' operation.  If IPC page fault
modeling is activated then arbitrary transfers will cause a page
fault in the client or the receiver. Our verification runs show
that the IPC system fully recovers from such faults.

A full verification run of this test takes about 41 MB of memory.
Interestingly, it takes longer to compile the test than it does to run
it.  It generates 94,000 states, and only executes to a maximum depth of 1650.
These are especially interesting results considering that over 60,000
lines of C code are generated by SPIN.
Because the send-receive is a fairly lockstep
operation--the server block until the client sends, the client blocks
until the server acks, etc--it is easy to see why there is little 
interleaving among the states.

This test does not cover other ``corner cases'' than the interleaving 
of page faults in the code and the recovery steps involved, in
particular it does not consider the case where either thread 
involved is canceled.

%%%%%%%% Ack
\subsection{The Ack Test}

This test, {\tt src/test/Ack-test.pr}, exercises the acknowledgment 
portion of the protocol.  Sending an ack tells the other half of your
connection that you have finished receiving and are ready to send data
back across the link.  

This tests much of the code exercised by the basic send test, in
addition the client send and ack-specific codes.

The scope of this test is similar to that of the basic send
test, it requires approximately 20 MB of memory to run. Again, the
compilation of the verifier takes longer than the verification itself.

This test has the same limitations as basic send, i.e., error 
conditions outside of a page fault are not considered.  Still, all
of the waiting, blocking and capturing occurs correctly in all
instances of page-faults and all possible arrival ordering. 

%%%%% Interposer
\subsection{The Interposer Test}

This is a more complicated test.  It involves three threads.  One is a
client whose server thread is a client to a third thread.   The
thread in the middle is acting as an interposer--it receives data from the
client and forwards it to the server.

This tests the middle thread's code path quite rigorously.  It is
acting as both a client and a server and can experience page faults on
either side of its connection.

As a more complicated test, this would require about 400 MB of memory
to hold all of its states.  (Its has about 10 million states.)  
Supertrace lets us run it in 130 MB of memory.

% Cancel-test.pr
\subsection{The Cancel Test}

This test, {\tt src/test/Cancel-test.pr}, was our ultimate goal for the
project.  We hoped to show that the exportable kernel state (which is
implemented via the thread cancel facility which allows a thread's
state to be extracted at any time) was robust.  It involves three
threads.  Two threads operate as client and server as in the basic
send test, endlessly connecting and sending data.  The third thread
sits in an infinite loop and cancels either the client or the server.  

While we haven't been able to run a complete successful verification
of this test, we still believe our model to be sound.  Currently
the only known deadlock is encountered 900 steps into a
verification. It is due entirely to a poor abstraction of ports and
port sets.  Specifically, we are not modeling the wait state, wait
queue and cancelability of threads that are waiting to connect to
their client or their server.  The initial abstraction we implemented
(a single bit for rendezvous) was too naieve.  The IPC code expects
the threads returned from a rendezvous to be in certain states.  It
also expects that the cancel pending flag will have been checked.  
This should be fixed as we continue to
maintain and expand the system for the Flux project.

This test is quite large. A complete state space exploration would
require on the order of 2 GB of memory.  The supertrace option in SPIN
lets it run in about 200 MB. The estimated coverage of the state
space is 99.9\%.  

\subsection{Other Tests}

We developed our Promela in a bottom-up fashion, starting with the
primitives used to implement the more complicated IPC operations.
We have a full suite of tests that test our specific abstractions--wait 
queues, condition variables, mutexes, the kentry layer, etc. 

\subsection{Test Analysis}

While debugging these tests, we have so far exclusively encountered 
bugs in our model of the system.  It is interesting to note that none 
of these bugs have been in the parts of the model that were derived
from the C code.  They have all been in those areas of the 
model where the C code could not be literally translated into Promela,
but where abstractions with equivalent semantics had to be reimplemented 
for efficiency and other reasons.  For example, a bug in
our model did not include checking for a pending cancel before blocking.  
This quickly broke the cancel test (i.e., within 100 steps.) 



% give some conclusions bla-bla-bla
% conclusions.tex - How we're wiser, better people now that it's over
%------------------------------------------------------------------------------

\section{Conclusions}
\label{conclusions}

The results from this project were fairly typical of a small
specification and verification effort: no major problems with the
system being modeled were discovered (but we have more confidence that
there are no problems), the specifiers gained insight into the
workings of a complex part of the system, and we now have valuable
documentation that should lower the learning curve for people trying
to learn about Fluke internals.

In addition generating the documentation, and providing a foundation
for further verification work, the project was successful in pointing
out one latent bug.  In implementing the Fluke security kernel, Steve
Smalley added a new security-related wait state, without realizing
that the wait states were organized in a hierarchy.  The hierarchy in
our documentation made him realize that the new wait state was wrong,
and that he had made a lucky choice.  So, the project has already had
success in eliminating some ambiguity.

We plan to integrate much of the documentation developed under this
project back into the Fluke kernel documentation and source code.  In
particular, Appendix A as well as the call traces contain information
that will be valuable to anyone studying the kernel.  We also have
many useful comments that will be integrated back into the source
code, such as the wait state hierarchy, etc.  The Promela code itself
may be valuable as encouragement for other OS teams to try out
verification projects.

The Fluke IPC verification effort has seen a great deal of conceptual
and organizational change in a very short amount of time.  In particular,
organizational techniques to promote Promela code understandability and
to keep the state space size under control saw much evolution.  Avenues
which initially seemed viable, clean and attractive proved in the end
to be difficult, unwieldy and self-defeating in often subtle and
unexpected ways.

We've attempted to present in the preceding sections the organization
that we currently find workable, and our optimism for its continued
viability.  We now seek to more fully justify this organization in
comparison with other approaches.  We first examine work by others,
and discuss improvements we feel have been made over past approaches.
We then move on to describe specific failed organizations that were
attempted by this effort, and the problems we encountered.

%-- Related Work --------------------------------------------------------------

\subsection{Related Work}

Two other interesting efforts have been made to verify the correctness
of operating systems' IPC mechanisms.  We examine these briefly here,
paying particular attention to difficulties and inflexibilities in
these other approaches which have been specifically addressed by
the Fluke IPC verification effort.

\subsubsection{Harmony}

Cattel~\cite{Harmony-verify} performed a SPIN-based verification of
the Harmony operating system, a ``portable real-time multitasking
multiprocess operating system'' developed by the Canadian National
Research Council in the 1980's and early 1990's.  Cattel's effort
sought to verify properties of Harmony as a whole, and attempted
to model all kernel services.  Like our effort, Cattel was concerned
primarily with the interprocess synchronization mechanisms at work
during IPC.

While Harmony IPC is less complex than Fluke IPC, there are structural
similarities between the way Cattel structured his Promela and the
way our Promela is structured.  Cattel held IPC state in an array of
``task descriptors'' and pointed at them with their array index.  We
formalized this to the notion of the Reference, but the basic idea is
the same.  Also like our effort, Cattel generated his models by studying
the implementation code of a pre-existing system.  Thus, like our
verification, Cattel had strong reason to believe that he was proving
things about his particular implementation of the system, and not just
its abstract specification.

However, Cattel seems to have been mainly concerned with showing
freedom from deadlock, while we would like to consider stronger
guarantees such as correct state progression in the presence of
cancellation, as well as some sense of correct data delivery in this
environment.  Also, Cattel spends a great deal of effort analyzing
and minimizing his scenario-space, with the implication that he
would generate, by hand, a scenario for every possible interaction of
sender and receiver.  We seek to exploit SPIN's talents at exploring
the scenario-space for us.  We ultimately envision processes which
nondeterministically try all allowed high-level operations, so that
we are sure to explore all possible user errors, in addition to
reasonable interleavings of operations.

\subsubsection{RUBIS}

Duval and Julliand utilitized SPIN to model the intertask communication
facilities of the RUBIS microkernel~\cite{RUBIS-verify}.  Like our effort,
theirs sought to build and exhaustively test usage scenarios.  However,
their effort placed emphasis on formally expressing mappings between
implementation source code and Promela code via a collection of
transformation rules.  While these transformation rules do not appear to
have been used to perform {\it automatic} transformation of source code
to verifier code, the authors state their belief that such an automated
transformation tool would be of great value.

We, on the other hand, question the success that such a tool might find.
It is often the case that the verification of desired properties requires
the addition of extra "hidden" state to a model, which represents some
semantic state outside the implementation domain, but related to the
property of interest.  The ability to add this semantic state can make
demands on the overall structure of the verification code.  We have seen
cases where semantic properties we wish to expose are distributed
across much C code, for implementation reasons.  In cases like these,
it is unlikely that an automated tool would have the necessary intuition
to extract these distributed semantics in a meaningful way.

We feel that a more plausible approach is the formulation of a
meta-language, rich in explicit semantic content, from which both
%%	               ^^^^^^^^          ^^^^^^^
%% would we then need warning labels???
%
Promela code and implementation-language code could be automatically
generated.  We feel that such a meta-language could be made flexible
enough to facilitate the evolution of software by allowing experimental
manipulation of both semantic and implementation constructs, always in
the presence of easy verification of desired properties.

The RUBIS modeling project also made a distinction between {\it
abstract models} and {\it detailed models}.  Abstract models are
generated from high-level semantics defined in the system
specification and are useful for formalizing the system functionality
independently of implementation.  Detailed models, on the other hand,
are generated with the implementation of the system in mind, and can
be used to verify that the system provides the behavior defined in the
specification.

It is our belief that the rigidity of this distinction is unnecessary.
The ``sliding {\tt d_step}'' technique we employ for state-space
management creates a spectrum of models from the abstract to the
detailed.  Placing {\tt d_step}s around necessarily atomic segments in
a minimal fashion creates a pure detailed model.  Placing {\tt
d_step}s around high-level functions creates a pure abstract model.
Since these models share the same code, we can be sure that they
embody exactly the same semantics.  The precise implementation of
these semantics, once verified, can be abstracted away and the same
Promela code can be used as the basis for constructing the next layer
to be verified.

The RUBIS emphasis on abstract modeling at specification-time and detailed
modeling at implementation-time ignores the complex and recursive
interrelationship of these two processes.  We believe the ability to slide
{\tt d_step}s lower and formally understand the semantic impact of late-cycle
specification changes to be of great potential value to living software
projects whose goals are in constant flux.

%-- Lessons Learned -----------------------------------------------------------

\subsection{Lessons Learned}

The progress of the Fluke IPC verification project thus far fills us with
optimism.  However, we've traveled down our share of blind alleys, and
present brief summaries of these here:

\begin{itemize}

\item
We were originally going to encapsulate the state associated with
Thread objects as local state to a particular proctype, and access it
by sending message to that proctype.  The rationale was that the PO
reductions allowed by encapsulating the state and declaring the
communication channel "xs" to the "state-proc" would outweigh the
added state-vector length that would result from the extra proctype
instances.  However, we found it impossible to get appropriate
blocking semantics when locking Mutex'es held by this state-proc.
Also, since local variables were required to hold values "gotten" from
the state-proc (there is no stack), the state-vector costs were much
greater than our original estimates.  The approach was scrapped and
thread state was made global, greatly simplifying all code that
accessed it.

\item
\xxx{Any others?}

\end{itemize}

Also, we've noticed some potential improvements to the SPIN system itself:

\begin{itemize}

\item
SPIN's {\tt if} syntax is less than ideal for the task of translating nested
if-then-else statements.  Since "properly formatted" if :: :: fi statements
take several lines, and the ``:: else $\rightarrow$''  syntax requires deep
indentation, it is tedious to translate C structures of the form
{\tt if \ldots else if \ldots else if \ldots }

\item
SPIN has internal hard-coded limits on how much code can be in d_step
bodies.  We hit some constant hidden in {\tt d_step.c}, which we had
to increase from 512 to 1024 to get our code to compile.

\item
Variable names (not necessarily variable slots) and labels should be
scoped to their enclosing block, to allow goto's to be used inside
macros that are instantiated multiple times in the same proctype.

\item
SPIN should understand hexadecimal integer constants, like {\tt
0xfff0}.  Fluke ``wait states'' are even more cryptic in Promela than
they were in the C code, since they must be expressed as decimal values.
(Even the compilers generated by the 507 class do this!)

\item
Execution traces say "(1)" for macros whose bodies are skip, when the
macro name is more meaningful (for example enabling and disabling
interrupts).  This is a manifestation of the larger problem that cpp'ed
source is displayed in the traces, not "meaningful" source.

\item
A {\tt printf()} is not available that can print during a
verification. (Well, you can turn them {\em all} on, but if
you only want one, its impossible.)  This would be very
helpful in generating coherent run-time error messages.

\item
The {\tt assert()} should take a string argument explaining what went wrong.
It would make assert failures far more informative.

\item
More flexible pre-processing support should be standard.  In particular,
SPIN should honor the CPP environment variable, or at least run the
cpp on the user's PATH. (Currently the source hard codes {\tt /usr/lib/cpp}.)

\item
Most SPIN error messages are completely inane.  In particular, error messages
that refer to nonexistent proctypes, error messages that report the ASCII
values of trouble characters, and messages that say ``{\em thingX} seen near
{\em thingX}'' are particularly frustrating.

\item
Accurate reporting of error line number and source file would be useful.


\end{itemize}

Several of the above items result from a fundamental tension between
the simplicity that Promela's designers want you to focus on, and the
complexity of Fluke IPC.  Promela does not have functions, partly in
order to make designers think hard about making the specification as
simple as possible.  In order to adequately capture Fluke
functionality and keep the Promela code readable, we rely heavily on
a preprocessor to provide the illusion of function calls.  That made
debugging difficult, but the only other choice would be to duplicate 
code in many different places.

Finally, a note on the basic approach we took.  Although the use of
Promela/SPIN was effective in analyzing a reverse engineered
specification of the Fluke IPC path, we do not believe that SPIN would
have been as effective in developing a from-scratch specification.  It
is certainly possible to develop a specification (in a natural style
for Promela), but it is very unlikely that the corresponding
implementation could be so easily related back to the specification.
Also, the Promela specification is of limited value in developing
alternate (e.g., highly optimized) versions of Fluke IPC.

\subsection{Prospects for the Future}

In conclusion, we feel that the work completed thus far serves a dual
purpose.  First, it represents a successful proof-of-concept that
SPIN- based model-checking is a viable approach for verifying certain
properties about the Fluke operating system.  Second, it lays the
groundwork for the verification of many properties of the Fluke IPC
system.  In addition, we are optimistic that the techniques learned
thus far could be applied toward other facets of the Fluke operating
system.


%
\appendix
\section{Appendix}
This section contains details about the Fluke C implementation of the
IPC mechanism.

\subsection{Functions implementing components of reliable IPC
  paths}
 
The implementation makes use of a set of core functions
listed below.
Most paths are built from two phases: 1(send) and 2(receive).
All phase 1 routines are called with the current thread in an unknown
state, and on successful return, the other thread is captured as the
receiver. All phase 2 routines are called with the other thread
captured as receiver, and return with everything all done, ready to
return from the syscall. Routines that implement both phases 1 and 2
are marked _12_.

\subsection{Client-side reliable IPC path components}

\begin{itemize}
\item ipc_client_1_connect_send(client_thread, connected_ip)\\ 
  First disconnect any connection client_thread is involved
  in. Establish a
  connection to a new server thread using s_port_capture(). Set the
  server field of the client and the client field of the server
  appropriately and change the flags of the client and the server to
  show that the client is now a sender and the server is no
  longer a sender. Transfer the  minimum message from the client
  to the server. Set the IP field of the client to the connected_ip
  parameter and the IP field of the server to
  fluke_ipc_server_receive__entry. Then perform a reliable transfer
  operation from the client to the server.\\
%\xxx{explain and list the IPs somewhere in the sections on threads}\\

\item ipc_client_1_ack_send(client_thread, acked_ip)\\  
  First make sure the connection is active (by calling
  ipc_client_find_server). If the server is not waiting as a sender
  (i.e. it's wait state is not at least WAIT_IPC_SRV_SENDER) put the
  client thread to sleep in the WAIT_IPC_CLI_ASEND state and return. 
  If the server is waiting as a sender, capture the server. If the server
  wasn't waiting in WAIT_IPC_SRV_ORECV (i.e. it is not waiting for a
  over and receive) put the client thread to sleep in the wait state
  WAIT_IPC_CLI_ASEND, put the server on the ready queue (by calling
  thread_handoff) and return. If the above condition doesn't hold this
  means both the client and the server are ready to reverse. Change the
  flags of the server and the client to indicate that the client will
  now be a sender and that the server is no longer a sender. Transfer
  minimum message from the client to the server. Set the IP of the client
  to the acked_ip parameter and that of the server to
  fluke_ipc_server_receive__entry. Now perform a reliable transfer
  from the client to the server.
  
\item ipc_client_1_send(client_thread, out_wval)\\
  First make sure the connection is active (by calling
  ipc_client_find_server). If the server is not waiting as a receiver
  (i.e. it's wait state is not at least WAIT_IPC_SRV_RECEIVER) put the
  client_thread to sleep in the wait state WAIT_IPC_CLI_SEND and
  return. Otherwise capture the server as a receiver. If the server
  was waiting to receive (i.e. in the wait state 
  WAIT_IPC_SRV_RECV) do a reliable transfer from the client to the
  server. If the server was waiting in the wait state
  WAIT_IPC_SRV_ASEND, this means that the server is acking the client before
  it is done and so just throw away the send data.

\item ipc_client_2_over_receive(client_thread, wval)\\
%wval could be just IPC_SRV_RECEIVER or IPC_SRV_ASEND.. What is
%it semantically - the state of the server ?
%If the wval parameter is WAIT_IPC_SRV_RECV do a IPC_FINISH_RECEIVE
% \xxx{What does FINISH do}(Since the server still thinks it's
% receiving data notify it that the message has ended).
  If the wval parameter is WAIT_IPC_SRV_RECV, this means that the
  server still thinks it is receiving data and so notify it that the
  message has ended using a IPC_FINISH_RECEIVE.
  Put the client to sleep in the wait state WAIT_IPC_CLI_ORECV and put
  the server on the ready queue.\\

\item ipc_client_12_receive(client_thread)\\
  First make sure the connection is active (by calling
  ipc_client_find_server). If the server is not waiting as a sender
  (i.e. it's wait state is not WAIT_IPC_SRV_SENDER), put the client
  thread to sleep in the wait state WAIT_IPC_CLI_RECV (by calling
  thread_wait) and return. Otherwise capture the server thread as the
  sender. If the the server is not waiting to send (i.e. it's wait
  state is not WAIT_IPC_SRV_SEND, it means that the server is done
  sending; so release the server, notify the client that the message
  has ended using a IPC_FINISH_RECEIVE and return
  KR_RESTART. Otherwise put the client to sleep in the wait state
  WAIT_IPC_CLI_RECV and release the server.

\end{itemize}

\subsection{Server-side reliable IPC path components}
\begin{itemize}
\item ipc_server_1_ack_send(server_thread, acked_ip)\\
 Similar to ipc_client_1_ack_end() described above.

\item ipc_server_1_send(server_thread, out_wval)\\
 Similar to ipc_client_1_send() described above.

\item ipc_server_2_wait_receive(server_thread, wval)\\
 Break the connection with the client thread and release it. Nuke the
 client field of the server and the server field of the client. Do a
 FINISH_RECEIVE, set the IP to fluke_ipc_wait_receive__entry
 put the client on the ready queue and then 
 wait on a port set for an incoming IPC request.

\item ipc_server_2_over_receive(server_thread, wval)\\
 Similar to ipc_client_2_over_receive() described above.

\item ipc_server_3_wait_receive(server_thread)\\
 Set the IP of the server to fluke_ipc_wait_receive__entry and wait
 on a port set for an incoming IPC request.

\item ipc_server_12_receive(server_thread)\\
 Similar to ipc_client_12_receive() described above.
\end{itemize}

\subsection{IPC machine dependent macros}

{\tt EXC_SET_IP(ip)} and {\tt EXC_GET_IP()} are wrappers that set the
machine depenedent instruction pointer.  They are used for setting a
new kernel entry point.  For example, during a {\tt
fluke_client_connect_send_over_receive()}, after the ``connect'' phase
is complete, the kernel will set the IP to be the {\tt
fluke_send_over_receive()} entry point.  To exit the kernel, 
the IP to {\tt fluke_nop__entry}.

{\tt IPC_FINISH_RECEIVE(thread, status)}.  This macro sets the
user-return code for a kernel syscall, and sets the thread's
IP to {\tt fluke_nop__entry}.  It also does some silly stuff
with the ``min_msg'' registers--but only because they aren't
real registers.

{\tt IPC_STATUS(thread)} sets the status code for the current ipc
syscall.  The details are quite grungy:  On the x86 the eax register
is overloaded.  The bottom 16 bits are the number of buffers
the invoker is using (an in/out parameter), the top 16 bits are
the return code of the IPC call.

{\tt EXC_RETURN_INSANITY(cond)} set the {\tt exc_state.code} to
the given insanity condintion, and returns a KR_USER_EXCEPTION.

\subsection{IPC Pickle Operations}

\begin{itemize}

\item {\tt void ipc_pickle(s_thread_t *client, s_thread_t *server)}

Takes client and server threads.  They must be pointing to each other
respectively.  Moves the active pointers into indirect links.  Zero's
out the direct pointers.
 
\item {\tt void ipc_unpickle(s_thread_t *client, s_thread_t *server)}

Destroy the indirect pointers.  (We don't have to ``follow'' them,
because they were passed in as parameters.)

Check that the two threads agree on the ``direction'' of the
connection--

%\xxx{What do the FLUKE_THREAD_/FOO/_SENDER bits mean?}

%\xxx{Shouldn't the code assert() that the passive links are correct?}
%% Answer: Can't need to, as only called after they're nuked.

The server's direct client pointer is set to the client and
vice-versa-vice.

\item {\tt int ipc_client_unpickle(s_thread_t *client, int no_wait)}

Client must be the current thread.  It must have no direct server
pointer.  (Its trying to restore the direct pointer.)

First, the indirect server link is turned into a direct pointer. If
the link is bogus, 0 is returned.  (The object is locked (the {\tt sob}
lock) if the link was successfully followed.)  If the link pointed to a
non-thread object, then unlock the object and return.

%\xxx{Couldn't server be waiting for _its_ server to pickle itself?}

If the server is in WAIT_IPC_UNPICKLE, meaning that it is waiting
for its ``other half'' to pickle a connection,  then we unlock the
server.  The thread_wakeup() call put the thread in the WAIT_NONE
state, and its not on any queues, so we have it ``captured''.

If the captured server's {\em passive client link} isn't pointing at
us, then the client blocks in WAIT_IPC_PICKLE and hands off to the
server.  (An exception being, that if the no_wait flag is passed in
non-zero, then the server is readied and we return immediately without
blocking.) Otherwise, call {\tt ipc_unpickle(client, server)} which
will restore the active links.  Then ready the server, as we had it
captured.

But, if the server wasn't in WAIT_IPC_UNPICKLE, then if we're the
no_wait flag was passed in non-zero, 

\item {\tt int ipc_server_unpickle(s_thread_t *server, int no_wait)}

\end{itemize}

\subsection{Thread Wakeup and Ready Functions}

These are the functions used to put threads to sleep, wake them
up and manipulate their wait state.

\begin{itemize}
\item{\tt thread_wakeup(target_thread, required_state)}

Disables interrupts, spin locks 'target_thread's wait_state lock
and then calls {\tt id_thread_wakeup_locked} which wakes up and
captures 'target_thread' iff its current wait value is
{\em at least} 'required_state'. (See the Wait State section for
details on the wait states.)

If 'target_thread' is waiting in {\tt WAIT_ON_COND} state, then it is
removed from the queue and returned to the WAIT_NONE state.  The old
wait state of the target thread is returned.
(Note that there is a potential race condition handled in here, too.
If a thread is woken from a condition variable wait, the kernel
can't change its wait state until after its been removed from the
wait queue.  Thus in this code, if the thread is waiting on a
condition variable, but is not actually on the specified list, then
it is in the process of being removed by {\tt id_cond_wakeup()} which
has removed it from the list, but has not updated its state.)

If 'target_thread' is not in {\tt WAIT_COND} then 'target_thread's
old wait state is returned and it is put into the WAIT_NONE state.

If 'target_thread' is not in at least 'required_state' then 0 is
returned.

\item{\tt thread_handoff(current_thread, state, target_thread)}

Give control to 'target_thread' atomically,
put ourselves in waiting 'state', and handle the possibility of
us getting canceled.

Disable interrupts, acquire spinlocks on both wait states in defined
order to avoid deadlock, make sure that our own wait state is
\texttt{WAIT_NONE}. Then, put 'target_thread' on the ready queue
using \texttt{id_thread_ready} (because it's locked), release
the lock and let the other thread go. That's all that's done to
the other thread.

At this point, a cancel may be already pending (indicated by
\texttt{wait_state.cancel_pending}). If one is pending, then there are
two cases:
\begin{itemize}
    \item The state we want to wait in is {\em not cancelable} (i.e.,
        it doesn't have \texttt{WAIT_CANCELABLE} in its pattern - unlike
        most of the intermediate IPC states) - then set our resume_rc
        to \texttt{KR_CANCEL}. Note that we don't return here.
        This is a somewhat special case.
    \item If, however, the state we should wait in is {\em cancelable},
        then unlock the current thread object, enable interrupts again
        and return immediately with \texttt{KR_CANCEL}.
        Don't context switch. Consider yourself canceled.
\end{itemize}
At this point, resume_rc is either set to \texttt{KR_RESTART} or
to \texttt{KR_CANCEL}, depending on whether we have a cancelation
pending or not. We now set the current thread's wait state to 'state'
and go to the dispatcher.

Cryptic comment \#21a in the code says:
\begin{verbatim}
/*
 * We don't need to add the current thread to any wait queue,
 * because the wait state itself defines the "ownership" of the thread.
 */
\end{verbatim}
This simply means that another thread will explicitly point at us and
make us ready - rather than we being one thread among dozens waiting
for an event to happen.

Eventually, the dispatcher will reschedule us (because someone else
put us on the ready queue), and then we reenable interrupts and return
with either \texttt{KR_RESTART} or \texttt{KR_CANCEL}, depending on
whether a cancelation was pending or not when we invoked the
dispatcher.

Note that the ``unusual'' case, where the dispatcher is invoked
even though we already knew a cancelation was pending is only used
when actually handling a cancelation.

\item{\tt thread_wait(current_thread, state, other_thread, other_state)}

Put current_thread to sleep in the given
state, (like thread_handoff) unless the other_thread is already in
other_state.  Perhaps perform an atomic action on other_thread before
we go to sleep.

This function is similar to thread_handoff() above (80\% of the code is
identical), but instead of simply handing off to another thread, it
checks to see if other_thread has reached other_state.  If so,
thread_wait returns immediately with \texttt{KR_RESTART}.  Otherwise,
the current_thread is put to sleep in wait_state state.

The function first disables interrrupts, and grabs spin locks for both
of the threads' wait_states.

If other_state is equal to 0, the other_thread is canceled before
current_thread is put to sleep.  Also, if other_state is equal to -1,
it is a magic flag to unlock the other_thread\verb\->\sob.lock.  Both of these
actions must be performed atomically due to potential race conditions
with the other_thread waking up before current_thread is really asleep.

At this point, a cancel may be already pending (indicated by
\texttt{wait_state.cancel_pending}). If one is pending, then there are
two cases:
\begin{itemize}
    \item The state we want to wait in is {\em not cancelable} (i.e.,
        it doesn't have \texttt{WAIT_CANCELABLE} in its pattern - unlike
        most of the intermediate IPC states) - then set our resume_rc
        to \texttt{KR_CANCEL}. Note that we don't return here.
        This is a somewhat special case.
    \item If, however, the state we should wait in is {\em cancelable},
        then unlock the current thread object, enable interrupts again
        and return immediately with \texttt{KR_CANCEL}.
        Don't context switch. Consider yourself canceled.
\end{itemize}
At this point, resume_rc is either set to \texttt{KR_RESTART} or
to \texttt{KR_CANCEL}, depending on whether we have a cancelation
pending or not. We now set the current thread's wait state to 'state'
and go to the dispatcher.

Cryptic comment \#21a in the code says:
\begin{verbatim}
/*
 * We don't need to add the current thread to any wait queue,
 * because the wait state itself defines the "ownership" of the thread.
 */
\end{verbatim}
This simply means that another thread will explicitly point at us and
make us ready - rather than we being one thread among dozens waiting
for an event to happen.

Eventually, the dispatcher will reschedule us (because someone else
put us on the ready queue), and then we reenable interrupts and return
with either \texttt{KR_RESTART} or \texttt{KR_CANCEL}, depending on
whether a cancelation was pending or not when we invoked the
dispatcher.

Note that the ``unusual'' case, where the dispatcher is invoked
even though we already knew a cancelation was pending is only used
when actually handling a cancelation.

\item{\tt thread_cancel(thread)}

Set the cancel pending bit in thread.
Disables interrupts, locks the wait_state.lock, and sets
thread.cancel_pending. If it can wakeup the thread, it does
so, sets the thread's resume_rc to KR_CANCEL and readies it.

\end{itemize}


\bibliography{report,sys}
\bibliographystyle{abbrv}

\end{document}

