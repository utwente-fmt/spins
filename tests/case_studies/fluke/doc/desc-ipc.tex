\subsection{Overview of Fluke IPC}

This section gives the basics of the Fluke reliable IPC mechanism.
Two other forms of IPC in Fluke, One-way and Idempotent, are
described in \cite{Flukedocs:96}.

{\em Reliable IPC} is the mechanism Fluke provides
for general, high-performance, reliable, protected communication.
Reliable IPC in Fluke operates in a stream-oriented model,
in which two threads connect to each other directly
and can send data to each other as if through a pipe.
These connections are symmetrical but half-duplex,
meaning that data can only flow in one direction at a time.
However, the direction of data flow can be reversed
without destroying the connection.

Any Fluke thread can be involved in two reliable IPC connections at once:
any thread may have both a connection to a {\em client}
thread--a thread that ``called'' it,
as well as a connection to a {\em server} 
thread--a thread that it has ``called''.

Either side of a reliable IPC connection
may be held up arbitrarily long while waiting for the other side.
%e.g. because of page faults in the other task that need to be handled.
Thus, there are no guarantees regarding the ``promptness''
with which a server thread can perform its work and disconnect.
A server thread is bound to a particular client thread
for the duration of the IPC operation.

Communication within the kernel is based on unilateral
synchronization--interacting threads are required to cooperate with
each other.
Threads effectively ``ask'' other threads to complete an operation
on their behalf, blocking until the result is available.  For example,
during a standard reliable IPC operation, the server thread will
block waiting for a client to show up.  It blocks in 
a state labeled as ``waiting to receive''.  The client will 
grab the server and set up the connection.  Once the connection is
established, it changes the {\em server's} state to reflect an
active connection (just in case they get disconnected.) and then
starts transferring data to the server.  When the transfer is 
completed, the client puts the server back on the ready queue so
that it may run.  When the server wakes up it will restart as
if it had been doing a receive operation. The client will go
to sleep in a state indicating that it has finished a send.  The
server wakes up and notices the client is in the right state,
and then restarts the client, and returns to the user itself.

It is important to note that while the 
thread is blocked in a specific ``wait state'' it will not
be restarted until its counterpart in the reliable IPC operation
restarts it.

The IPC implementation uses {\it wait state values} to keep track of
the state in which a particular thread is waiting. For example, if a
thread is currently a server (in an IPC connection) and is waiting to
send data, its wait state value will be {\tt WAIT_IPC_SRV_SEND}.
The set of wait states forms a tree whose
leaves are single states. The leaves are represented in a binary
encoding such that a single bit test suffices to determine whether
a state belongs to a certain subtree, i.e. is a member of a subset of 
states.   See Figure~\ref{waitStates.fig} for the tree of valid
wait states.

\begin{figure}[th]
{\small
\begin{verbatim}
WAIT_NONE
WAIT_ANYTHING ->
     WAIT_IPC_PICKLE
     WAIT_STOPPED
     WAIT_CANCELABLE ->
             WAIT_ON_COND
             WAIT_IPC_SRV_DSCN
             WAIT_IPC_SRV_RECEIVER ->
                     WAIT_IPC_SRV_RECV
                     WAIT_IPC_SRV_ASEND
             WAIT_IPC_SRV_SENDER ->
                     WAIT_IPC_SRV_ORECV
                     WAIT_IPC_SRV_SEND
             WAIT_IPC_CLI_DSCN
             WAIT_IPC_CLI_RECEIVER ->
                     WAIT_IPC_CLI_RECV
                     WAIT_IPC_CLI_ASEND
             WAIT_IPC_CLI_SENDER ->
                     WAIT_IPC_CLI_ORECV
                     WAIT_IPC_CLI_SEND
             WAIT_IPC_UNPICKLE
             WAIT_IPC_IDEMPOTENT ->
                     WAIT_IPC_EXCEPTION
                     WAIT_IPC_CALL
                     WAIT_IPC_SECURITY
\end{verbatim}
} %end \small
\caption{Valid wait states for blocked threads}
\label{waitStates.fig}
\end{figure}


% XXX keep?
\subsection{Promela Implementation of Fluke IPC}

The Promela model of the IPC code resembles the implementation very closely. 
% Pat, I don't like the phrase "have been abstracted" or even 
% "have been abstracted away"
Very few details have been discarded or replaced with implementations
of equivalent abstract semantics. 
The majority of functions which model the reliable IPC path are 
implemented by directly translating the corresponding C functions to 
Promela. (The C functions are described in some detail in the Appendix.)
For example, the following is a snippet of C that implements the
{\tt sys_ipc_client_connect_send_over_receive()} function--the
kernel interface for that function.

\begin{verbatim}
        struct s_thread * const client = current_thread();
        int rc;

        rc = ipc_client_1_connect_send(client,
                                       fluke_ipc_client_send_over_receive__entry);
        if (rc) return rc;

        return ipc_client_2_over_receive(client, WAIT_IPC_SRV_RECV);
\end{verbatim}

Differing naming conventions aside, it can be seen that the 
``equivalent'' Promela looks almost identical:

\begin{verbatim}
        ipcClient1ConnectSend(currentThread,
                              threadExcConnectArg(currentThread),
                              ENTRYPOINT_CLIENT_SEND_OVER_RECEIVE);
        if 
        :: rc==0 ->
             ipcClient2OverReceive(currentThread, WAIT_IPC_SRV_RECV);
             /* sets rc, RETURN rc */
        :: else ->
             /* RETURN non-zero rc */ skip;
        fi;
\end{verbatim}

Since we don't model ports and port sets, the id of the
server thread is passed as a parameter (this is accomplished by
{\tt threadExcConnectArg(currentThread)}).

% --this seems obsolete
% There was one area of the code that was easy and worthwhile to
% simplify.  The data transfer.

\subsubsection{Transfer of data}
In the C implementation, the function {\tt ipc_reliable_transfer()}
copies the data and references from the sender space to the receiver space.  
This functions implements scatter/gather buffers, the bookkeeping
needed for reference transfer, and other things -- all of which are
outside the scope of our verification. 

In our Promela implementation, we simplify the data transfer to its
essence. Only a single byte of payload (which is part of the thread's 
exception state) is transferred from the sender to the receiver. 
%This is made use of in assertions
%which check if the data received is the same as the data that has been
%sent.

The possibility of either the sender or the receiver page faulting
during the transfer is implemented using the non-determinism that
Promela provides.  Note that the payload is copied only in the case in
which there is no error.  Additionally, because transfer is isolated to 
a single function we are able to make the inclusion of page fault
a configurable option.

