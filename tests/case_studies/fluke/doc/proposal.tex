% %%%%%%%%%%%%%%%%%%%%%%%%%%%%%%%
%
% Patrick Tullmann, John McCorquodale, Ajay Chitturi,
% Jeff Turner, and Godmar Back
%
% Fluke Verification Project
%
% %%%%%%%%%%%%%%%%%%%%%%%%%%%%%%%

% Document setup info
\documentclass{article}
\usepackage{times}
     
%%%%%%%
%%%%%%% Some standard \def's 

\long\def\com#1{}
\long\def\xxx#1{{\em {\bf Fix: } #1}}

% We don't need no steenkin' equations - just gimme a working underscore!
\catcode`\_=\active 

\def\psinc(#1,#2)#3{
\setlength{\unitlength}{1in}
{\unitlength =1pt
\centering\begin{picture}(#1,#2)
\put(0,0){\special{psfile=#3.eps}}
\end{picture}}}

\def\psfig(#1,#2)#3#4{
\begin{figure}[t] % XXX fix this, don't want at top
\psinc(#1,#2){#3}
\caption{\footnotesize #4}
\label{#3}
\end{figure}}

%%%%%%%
%%%%%%%

\topmargin 0pt
\advance \topmargin by -\headheight
\advance \topmargin by -\headsep
\textheight 8.9in
\oddsidemargin 0.3in  
\evensidemargin \oddsidemargin
\marginparwidth 0.5in
\textwidth 6in

%% I'm going to try to structure this like a paper, to 
%% minimize re-writing if things go well and this becomes
%% a real paper.

\title{\Large Verifying the Fluke $\mu$kernel IPC path}

\author{Ajay Chitturi~~~~John McCorquodale~~~~Patrick Tullmann~~~~\\
	Jeff Turner~~~~Godmar Back \\[2ex]
	{\sc CS 611--Formal Methods for System Design}
	}

%\date{October 14, 1996}
\date{November 12, 1996}

%BEGIN DOCUMENT--------------------
\begin{document}

\maketitle

{\it We propose to model Fluke reliable Interprocess Communication (IPC)
and verify 
first, that under normal execution the threads involved
will not deadlock.  Also, we will verify
that any IPC can be canceled at arbitrary points, without
leaving the threads involved deadlocked or lost.
We will also generate a descriptive formal model of the IPC
implementation in Fluke and the objects related to it.
}

\section*{Introduction}

Fluke~\cite{Flukedocs:96} is a microkernel in the tradition of Mach, QNX, and KeyKOS.
Internally, most operations in fluke are implemented as operations
on distinct kernel objects.  For example, in an IPC 
operation between two threads (client \& sever), when  
the client decides to send some data to the server it will
find the server thread kernel object, lock part of it, 
copy the data, adjust state in the server object to 
reflect the copied state, and then restart the server
on its way.

%% XXX this is _way_ too detailed.

One of the key properties of Fluke is that the state of all
kernel-mediated objects is extractable at any time.  This
extracted state is guaranteed to be consistent, requiring
that all operations on these objects appear to be atomic.
The kernel accomplishes this by cleanly ``rolling back''
non-atomic operations that get interrupted.

% XXX Flesh this example out more, make it real.
For example, a thread in Fluke may attempt to acquire the
``state'' of another thread (the current register values,
communication connection information, etc.)  This action
implicitly stops the targeted thread, and extracts the
required information.  
If the target thread is involved in a complex operation
(i.e. a large IPC data transfer,) the kernel may
have arbitrary internal state to keep track of the
transfer.  Instead of trying to store this arbitrary state
in the thread, it can roll the operation back to a ``clean''
point in the execution.  
At this point, the state can be taken and given to the
requesting thread.

\section*{Related Work}

\subsubsection*{Cancelable IPC operations}

The fact that Fluke claims to be able to cancel IPC operations at any
arbitrary point is non-trivial. 
In Mach, certain IPC operations are not cancelable at arbitrary points
(despite hard work to the contrary).  In Mach, a message send containing
multi-page out-of-line data cannot be interrupted and safely restarted.  
Amoeba has similar restrictions.   

\subsubsection*{Modeling Operating Systems in Promela}

The Harmony operating system is a ``portable real-time multitasking
multiprocessor operating system'', developed by the Canadian National
Research Council in the 1980's and early 1990's.  It is
microkernel-based, and provides interrupt management, scheduling, IPC,
memory management, and I/O management~\cite{Harmony-verify}.

The Harmony verification effort performed an exhaustive verification
of the IPC features of Harmony, focusing on deadlock and livelock.
The Harmony effort is similar to our own in that they developed their
detailed models through study of the Harmony code, resulting in very
low-level models that accurately reflect the details of the kernel.

We will be able to use many of the lessons learned from the Harmony
effort.  For example, they concluded that an abstract model was not
particularly useful, because there was not enough detail to prove 
absence of deadlock.  The paper also contains several tips for
managing the state space problem.  

\com{
Our effort differs from that of the Harmony modeling in the extent to
which we model the interaction of the IPC system with other parts of
the Fluke kernel.  In particular, instead of a complete model of the
IPC subsystem, we choose only one of the types of IPC, and examine the
guarantees that can be made about operations on thread objects.
}

The RUBIS operating system is another microkernel-based operating
system, developed by a company called TELEMAT in Switzerland.  The
paper~\cite{RUBIS-verify} 
claims to have performed an exhaustive verification of the IPC
features of RUBIS.  However, they are vague about what
properties are actually verified (e.g., ``Different kinds of errors
have been detected''), and the claim is not substantiated.

\com{
As with the Harmony effort, our project differs from the RUBIS
verification in that we intend to do more than just prove that there
are no errors in the IPC system.
}

It is important to note that both of these efforts used Promela and
SPIN to model their system.

\section*{Fluke Interprocess Communication}

This section gives the basics of the reliable Fluke IPC mechanism.
Two other forms of IPC in Fluke, One-way and Idempotent, are
described in the glossary.

{\em Reliable IPC} is the mechanism Fluke provides
for general, high-performance, reliable, protected communication.
Reliable IPC in Fluke operates in a stream-oriented model,
in which two threads connect to each other directly
and can send data to each other as if through a pipe.
These connections are symmetrical but half-duplex,
meaning that data can only flow in one direction at a time.
However, the direction of data flow can be reversed
without destroying the connection.

Any Fluke thread can be involved in two reliable IPC connections at once:
any thread may have both a connection to a {\em client} thread,
or a thread that ``called'' it (its {\em caller}),
as well as a connection to a {\em server} thread,
or a thread that it has ``called'' (its {\em callee}).
\com{
These two connections are typically used to form a chain of threads
representing a {\em call stack}.
When a thread makes an invocation to a server,
the server thread is added to the top of the call stack;
when the server thread completes its work and breaks the connection,
it is removed once again from the call stack.
}

Either side of a reliable IPC connection
may be held up arbitrarily long while waiting for the other side.
%e.g. because of page faults in the other task that need to be handled.
Thus, there are no guarantees regarding the ``promptness''
with which a server thread can perform its work and disconnect;
the server thread is essentially dedicated to a particular client thread
for the duration of the IPC operation.

There are two objects supported by the Fluke kernel 
which are used to establish IPC connections, {\bf Ports}
and {\bf Port Sets}.

Ports are Fluke objects representing targets for IPC requests.
Ports might represent CORBA or MOM objects, for example,
or in Fluke implementations implementing full protection,
they may be used as secure capabilities.
A single server may maintain a number of ports,
each distinguishing a particular object, client,
or object/client pair.

Port sets are objects that act as a rendezvous point
between client threads attempting to perform IPC to a port
and server threads waiting for incoming IPC requests on one or more ports.
Multiple ports can be attached to a single port set,
and multiple server threads can wait for incoming requests
on a single port set at once.
This allows a single ``pool'' of service threads
to service incoming IPC requests from many ports
(i.e. requests associated with many ``objects'' the server exports).

Note that our model assumes that threads to connect to for IPC
have been ``found'' (the purpose of ports, port sets, and references).
Thus, we will not be implementing models of ports or port sets.

\section*{Modeling}

To accurately model the Fluke IPC implementation we propose
to develop Promela code that will follow Fluke's {\em implementation}
of IPC. 

The Fluke implementation is decomposed into layers and
objects.  Layers start from the user-level kernel interface and
proceed downward to more and more machine-oriented interfaces.
The objects encapsulate implementation details and provide
interfaces at the various layers.  This clean code layout
should make our job much simpler.

There are several ``primitive'' objects that are used
extensively throughout the IPC implementation.  
We plan to model and test each independent of the IPC
tests.  In addition to supporting our proposal,
this should provide a foundation for
modeling and verification of other parts of the Fluke kernel.

Part of our verification effort will be devoted to
formally describing these objects.   
This will facilitate generation of Promela code
modeling these objects.

\com{
Things to model
        - Mutex
        - Condition variables
        - Links (inter-object references)
        - Ports/Port Sets
        - Thread objects
}%com

The two most basic objects are the {\bf mutex} and
{\bf condition variable}.  
Both have very clean, small,
well defined interfaces; 
both objects are used throughout the kernel
to synchronize threads and protect kernel data.

The kernel provides {\bf links} which represent 
inter-object references.  They are used for
encapsulating pointers and for reference counting.
Initially we will model
these as simply as possible; we may add
verification of the reference counting later.

Two objects, {\bf ports} and {\bf port sets} are
used as rendezvous-points in the IPC 
(as described above).  We will assume that 
a thread has a direct handle on the thread it
is communicating with, as such ports and port
sets will not be modeled in our project.

The most important object in our model will be
the {\bf thread} objects.  These are the 
initiators of all IPC and cancel operations that
we will be interested in.  Again, we will only
model the portion of a thread object that is
relevant to IPC.  This distinction should be
quite simple because the Fluke thread objects
are compartmentalized quite well already.

To successfully model the 
IPC implementation, we will have to come up with
a scaled-down model of the actual code path.
We plan to generate logical control flows directly
from the C source using existing tools.  
These flows will help us determine exactly what
we want to model.

%The Setup
In order to verify the cancelability of Fluke IPC
operations we intend to simulate three running
kernel threads involved in a reliable IPC connection.
One thread will be a ``client'' another the ``server''
and the third thread will interpose on the connection.

\psfig(421,224){threads}{
	Three threads communicating as client, server, and
	interposer.  The fourth thread issues 
	{\tt fluke_thread_cancel()} operations on
	the others
	at arbitrary times.
}
	
A fourth thread, the ``rogue canceler'', will issue
{\tt fluke_thread_cancel()} operations to any
of the threads involved in the IPC at arbitrary points
in time (see Figure~\ref{threads}.)

We intend to show that despite the fact that the cancel
thread is canceling complex IPC operations, the thread
state and internal kernel state stay consistent and
none of the involved threads deadlock.

\section*{Verification}

While the descriptive model-generation phase of this project is sure
to generate many fine-grained verifiable properties, we 
specify four high-level properties that the Fluke Reliable
IPC system should exhibit.

First, and of most importance to the user, the IPC system should
be proven to provide reliable, correct data delivery to user processes,
or to fail in a well-defined and expected way, under all circumstances.
No allowed "successful" execution path should result in delivery of
partial, out-of-order, or otherwise corrupted data.  Furthermore,
no execution path defined in the descriptive phase to be "safe" (ie:
that does not require special user-guarantees) should result in
deadlock or lack of progress.

Second, thread state as visible to the user via the {\tt fluke_thread_get_state()}
function should be consistent and appropriate under all "safe" execution
paths.  These paths will include cancellations of IPC operations at
arbitrary points.  What constitutes "consistent and appropriate state"
will be defined during the descriptive model-generation process.
If the set of primitive thread operations can be
shown to exhibit "correct" behavior under all circumstances, it 
should follow that higher-level functionality is also correct.  
For example, we propose that if {\tt fluke_thread_cancel()} can be shown
to be correct, we can conclude that {\tt fluke_thread_get_state()}
is also correct.
Proof of any such propositions will be provided.

Third, thread cancellations occurring at arbitrary points must
not only preserve liveness properties, but also in some circumstances
promptness properties, as determined in the descriptive modeling
phase.

Lastly, thread cancellations must be shown to provide semantically
proper behavior whenever they can legally be applied.

\section*{Results}

We hope to accomplish several things with this project (beyond passing
the course, that is.)

\begin{itemize}
\item We will have a model of the IPC system.  In addition to having five more
Flux 
team members who are intimately familiar with the Fluke reliable IPC 
mechanism, this effort will provide clear documentation for new team
members.

\item We will probably be able to add stronger assertions to the
kernel code, derived from invariants we come up with during
the verification process.

\item We will be able to precisely define what Fluke guarantees
in terms of IPC.

\item We will have a strong infrastructure for verifying other
parts of the kernel implementation.

\end{itemize}

Lastly, there should be five more people with a comprehensive
understanding of formal verification methods as applied to
existing systems.

\com{
\section*{Future Work}

Interaction of Fluke Reliable IPC and DSM protocols.

Verification of memory management.

User-level server/nester verification.

Integration of the verification code/doc with the
kernel source.
}
\section*{Glossary}

\begin{description}
  \item[One-Way IPC] An unreliable one-way message send.  No
guarantees are made about the delivery of the message, and the sender
does not wait for any type of reply.

  \item[Idempotent IPC] Two-way communication, with ``at-least-once''
semantics.  The sender waits for a reply, and may have to resend until
a reply is received.  Duplicate messages must be tolerable at the
receiver side.

  \item[Reliable IPC] Two-way communication, with ``exactly-once''
semantics.  In Fluke, reliable IPC connections are half-duplex,
meaning that messages can be sent in only one direction at a time.

  \item[Cancel (a.k.a. ``stun'')] A cancel or stun operation is an
interruption of one thread by another, usually done using 
the {\tt fluke_thread_cancel()}
call.  The Fluke kernel will either allow the stunned thread to
continue execution to the end of an atomic block, or will roll the
stunned thread back to an appropriate (clean and consistent) point.

  \item[Rollback] In order to maintain consistent state when an
operation is interrupted, the Fluke kernel sometimes ``undoes'' the
operation, back to a point at which it can be safely restarted at a
later time.

  \item[Livelock] A system state in which no meaningful progress
is taking place.

  \item[Thread] A thread is the basic unit of execution in the Fluke
model.  It executes a particular flow of control for an application.
Multiple threads can be active at once in an application.

% XXX atomic

% XXX ``state''

\end{description}


\bibliography{sys,formal}
\bibliographystyle{abbrv}

\end{document}
